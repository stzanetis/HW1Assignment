\documentclass[a4paper,12pt]{report}

\usepackage[english,greek]{babel}   % For Greek and English language
\usepackage{kmath,kerkis}           % Kerkis font
\usepackage{listings}               % For adding C++ code
\usepackage{xcolor}                 % For custom colors
\usepackage{graphicx}               % For including images
\usepackage{amsmath}                % For advanced math formatting
\usepackage{hyperref}               % For creating hyperlinks
\usepackage{geometry}               % To set page margins
\usepackage{subfig}                 % For subfigures
\usepackage{titlesec}               % For customizing chapter titles
\geometry{a4paper, margin=1in}      % Set page margins

\titleformat
{\chapter}
[block]
{\normalfont\bfseries\huge}
{\thechapter\quad}
{0.5em}
{}
[
\vspace{1ex}
\noindent\rule{\textwidth}{0.4pt}
]

\lstdefinestyle{verilog}{
    language=Verilog,
    basicstyle=\ttfamily\small,                     % Use small monospaced font
    keywordstyle=\color{blue},                      % Blue for keywords
    commentstyle=\color{green},                     % Green for comments
    stringstyle=\color{red},                        % Red for strings
    morekeywords={module,endmodule,always,if,else}, % Additional Verilog keywords
    numbers=left,                                   % Line numbers on the left
    numberstyle=\tiny\color{gray},                  % Style for line numbers
    stepnumber=1,                                   % Line numbers every 1 line
    numbersep=5pt,                                  % Distance between numbers and code
    backgroundcolor=\color{white},                  % Background color for code block
    showstringspaces=false,                         % Don't show space in strings
    frame=single,                                   % Frame around the code
    captionpos=b,                                   % Position of caption (bottom)
    breaklines=true,                                % Automatically break long lines
    breakatwhitespace=true,                         % Break lines at whitespace
}

\def\tl{\textlatin}
\def\tg{\textgreek}

\begin{document}

\title{\textbf{Ψηφιακά Συστήματα \tl{Hardware} σε Υψηλά Επίπεδα Λογικής 1} \\ Αναφορά Εργασίας}
\author{\textbf{Σάββας Τζανέτης} \\
\textbf{10889} \\
\href{mailto:empty@auth.gr}{\tl{stzanetis@ece.auth.gr}}}
\maketitle

\tableofcontents

\chapter*{Εισαγωγή}
    \large Αυτή η αναφορά είναι μέρος της εργασίας του μαθήματος \textbf{Ψηφιακά Συστήματα \tl{Hardware} σε Υψηλά
    Επίπεδα Λογικής 1} του \textbf{Αριστοτέλιου πανεπιστημίου Θεσσαλονίκης}. Σε αυτή την εργασία, μας ζητήθηκε
    να υλοποιήσουμε μια αριθμομηχανή, καθώς και έναν επεξεργαστή \textbf{\tl{RISC-V}} με την βοήθεια των παρακάτω
    ασκήσεων σε γλώσσα \textbf{\tl{Verilog}}. Στην αναφορά θα γίνει μια σύντομη επεξήγηση των ασκήσεων, σχολιασμός
    του κώδικα και παρουσίαση των ζητούμενων διαγραμμάτων και κυματομορφών των προσομοιώσεων.
\chapter{Πρώτη Άσκηση}
\chapter{Δεύτερη Άσκηση}
\chapter{Τρίτη Άσκηση}
\chapter{Τέταρτη Άσκηση}
\chapter{Πέμπτη Άσκηση}
\end{document}